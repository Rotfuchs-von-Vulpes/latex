\documentclass{article}

%\usepackage[pages=all, color=black, position={current page.south}, placement=bottom, scale=1, opacity=1, vshift=5mm]{background}
%\usepackage[margin=1in]{geometry}
\usepackage{amsmath}
\usepackage{amsthm}
\usepackage{amssymb}
\usepackage[utf8]{inputenc}
\usepackage{hyperref}
\usepackage{natbib}
%\usepackage{biblatex}
%\addbibresource{blib.bib}

\title{Um simples teste de \LaTeXe}

\author{Rotfuchs von Vulpes}

\begin{document}
    \maketitle
    \begin{abstract}
        Este resumo não te diz nada, logo vem um monte de nada a seguir e leia por conta e risco. Continua lendo? Interessante, então vai ter que ouvir: Cientistas deviam estar no poder, pronto falei. A seguir você verá testes de formulas, acentos, ç, ü e outros caracteres especiais, formulas do tipo $2+2=4$ tabelas, imagens, figuras e um curto texto de opinião, tudo em \LaTeX, e não vou usar Lorem Opision porque isso é adimitir falta de criatividade. Emtão, aproveite. LUA é bom.
        
        \noindent\textbf{palavras-chave:} artigo, testes, matemática, computação, desenho
    \end{abstract}

    \tableofcontents

    \section{Introdução}
    \subsection{Premielinares}

    \section{Formulas}
    \subsection{introdução}
    \subsection{Lógica}
    Lógica Matemática é uma sub-área da matemática que explora as aplicações da lógica formal para a matemática. Basicamente, tem ligações fortes com matemática, os fundamentos da matemática e ciência da computação teórica.\cite{logica}
    \\Lógica é realmente muito Interessante, gosto bastante, mas ainda não sei calcular, mas acabei de aprender a fazer tautologias. Eis um exemplo de tautologia em lógica proposicional:
    \[
        ((p \lor q) \land (p \rightarrow r)) \lor (\lnot(p \lor q)) \lor (\lnot (p \rightarrow r)).
    \]
    Agora uma tautologia em lógica de primeira ordem:
    \[
        (((\exists x Rx) \land \lnot(\exists x Sx)) \rightarrow \forall x Tx) \leftrightarrow ((\exists x Rx) \rightarrow ((\lnot \exists x Sx) \rightarrow \forall x Tx)).
    \]
    \subsection{Conjuntos}
    Teoria dos conjuntos ou de conjuntos é o ramo da matemática que estuda conjuntos, que são coleções de elementos. Embora qualquer tipo de elemento possa ser reunido em um conjunto, a teoria dos conjuntos é aplicada na maioria das vezes a elementos que são relevantes para a matemática. A linguagem da teoria dos conjuntos pode ser usada nas definições de quase todos os elementos matemáticos.\cite{conjuntos}
    \\Conjuntos é Interessante, porem ele em si não é muito legal, mas quando munido de operações, isto é, um \textit{anel}, aí as coisas ficam interessantes.
    \[
        \mathbb{N} \subset \mathbb{Z} \subset \mathbb{Q} \subset \mathbb{R} \subset \mathbb{C} \subset \cdots
    \]
    \subsection{Funções}
    Uma função é uma relação de um conjunto $A$ com um conjunto $B$ . Usualmente, denotamos uma tal função por $f : A \rightarrow B$ ,  $y = f ( x )$, onde $f$ é o nome da função, $A$ é chamado de domínio, $B$ é chamado de contradomínio e $y = f (x)$ expressa a lei de correspondência (relação) dos elementos $x \in A$ com os elementos $y \in B$. Conforme suas características, as funções são agrupadas em várias categorias, entre as principais temos: função trigonométrica, função afim (ou função polinomial do 1° grau), função modular, função quadrática (ou função polinomial do 2° grau), função exponencial, função logarítmica, função polinomial, dentre inúmeras outras.\cite{funcoes}
    \\Funções são estremamente uteis, principalmente em materias de calculo e fisica. Gosto muito de funções polinominais, eis um exemplo:
    \[
        f(x)=x(x-3)(x+2)
    \]
    Fica como tarefa pra casa achar as raizes.
    \subsection{Álgebra}
    Álgebra é o ramo da matemática que estuda a manipulação formal de equações, operações matemáticas, polinômios e estruturas algébricas.[1] A álgebra é um dos principais ramos da matemática pura, juntamente com a geometria, topologia, análise, e teoria dos números.\cite{algebra}
    \\Esta é a minha área favorita da matemática, me considero um álgebrista! è realmente muito útil em problemas de matemática, computação e física. A minha parte favorita é isolar variáveis, aqui um sitema de passos:
    \[
        \begin{array}{rcl}
            y&=&\frac{x}{(x-1)(y+2)}\\
            y(x-1)(y+2)&=&x\\
            (xy-y)(y+2)&=&x\\
            y(xy-y)+2(xy-y)&=&x\\
            (xy^2-y^2)+(2xy-2y)&=&x\\
            xy^2+2xy-2y-y^2&=&x\\
            -2y-y^2&=&x-2xy-xy^2\\
            -2y-y^2&=&x(1-2y-y^2)\\
            \frac{2y+y^2}{2y+y^2-1}&=&x
        \end{array}
    \]
    \subsection{Geometria}
    A geometria (em grego clássico: $ \gamma \epsilon \omega \mu \epsilon \tau \rho \iota \alpha $ ; geo- "terra", -metria "medida") é um ramo da matemática preocupado com questões de forma, tamanho e posição relativa de figuras e com as propriedades dos espaços. Um matemático que trabalha no campo da geometria é denominado de geômetra.\cite{geometria}
    \\Outra área que amo muito, em particular trigonometria, e as provas costumam ser lindas, eis aqui a formula para a área de um triangulo equilátero de lado $l$:
    \[
        A=\frac{r^2\sqrt{3}}{4}
    \]
    \subsection{Probabilidade}
    A teoria das probabilidades é o estudo matemático das probabilidades. Pierre Simon Laplace é considerado o fundador da teoria das probabilidades.
    \\Os teoremas de base das probabilidades podem ser demonstrados a partir dos axiomas de probabilidade e da teoria de conjuntos.
    \\Os teoremas seguintes supõem que o universo $\Omega$ é um conjunto finito, o que nem sempre é o caso, como por exemplo no caso do estudo de uma variável aleatória que segue uma distribuição normal.

    \begin{enumerate}
        \item A soma das probabilidades de todos os eventos elementares é igual a 1.
        \item Para todos os eventos arbitrários A$_1$ e A$_2$, a probabilidade de os eventos se realizarem simultaneamente é dada pela soma das probabilidades de todos os eventos elementares incluídos tanto em A$_1$ como em A$_2$. Se a intersecção é vazia, então a probabilidade é igual a zero.
        \item Para todos os eventos arbitrários A$_1$ e A$_2$, a probabilidade de que um ou outro evento se realize é dada pela soma das probabilidades de todos os eventos elementares incluídos em A$_1$ ou A$_2$.
    \end{enumerate}
    
    As fórmulas seguintes exprimem matematicamente as propriedades acima:\cite{probabilidade}
    \[
        \sum_{\omega \in \Omega}P(\{\omega\})=P\left(\bigcup_{\omega \in \Omega} \{\omega\}\right)=1
    \]
    \[
        P[A_1 \cap A_2] = \sum_{\omega \in A_1 \cap A_2} P(\{\omega\})
    \]
    \[
        P[A_1 \cup A_2] = \sum_{\omega \in A_1 \cup A_2} P(\{\omega\})
    \]
    Eu não gosto muito de probabilidade, mas pretendo aprender, já que é extremamente útil. O maximo que sei é que dado evento $P$ com probabilidade $p$ de acontecer e outro evento $Q$ com probabilidade $q$ de acontecer é dado por:
    \[
        \omega = pq
    \]
    \subsection{Combinatória}
    A combinatória é um ramo da matemática que estuda coleções finitas de elementos que satisfazem critérios específicos determinados e se preocupa, em particular, com a "contagem" de elementos nessas coleções (combinatória enumerativa), com decidir se certo objeto "ótimo" existe (combinatória extremal) e com estruturas "algébricas" que esses objetos possam ter (combinatória algébrica).
    \\O assunto ganhou notoriedade após a publicação de textit{Análise Combinatória} por Percy Alexander MacMahon em 1915. Um dos destacados combinatorialistas foi Gian-Carlo Rota, que ajudou a formalizar o assunto a partir da década de 1960. E, o engenhoso Paul Erdős trabalhou principalmente em problemas extremais. O estudo de como contar os objetos é algumas vezes considerado separadamente como um campo da enumeração.\cite{combinatoria}
    \\Análise combinatória é interessante, e simples, talvez seja a área que mais deixo de lado, atrás apenas de matemática financeira, que aliás não tem tópico neste artigo.
    \\Suponhamos que você tem $n$ 
    \subsection{Calculo diferencial}
    \subsection{Calculo integral}
    \subsection{Calculo vetorial}
    \subsection{Geometria analitica}

    \section{Computação}
    \subsection{Opinião}
    A programação começou na minha vida como uma obrigação, em um curso de TI, mas logo me apaixonei por lógica da programação, linguagens de programação, e linguagens de marcação.
    \subsection{Minha experiência}
    \subsection{Pseudo-algoritmos}

    \section{Desenhos e figuras}
    \subsection{Desenhos}
    \subsection{Figuras em \TeX}

    \section{Conclusão}

    \medskip

    \bibliographystyle{unsrt}
    \bibliography{bib}
    
\end{document}